%QQQ Below: "10pt" prints in 10-point. "12pt" prints in larger 12-point.
\documentclass[12pt,letterpaper]{article}
\usepackage{amsthm,amssymb,amsmath}
\usepackage[pdftex]{hyperref}
%QQQ To print single-spaced, comment out the line below (precede it with a "%").
%QQQ To print double-spaced, remove the "%" at the beginning of the line.
%\renewcommand{\baselinestretch}{2}

% MARGINS

\setlength{\textwidth}{6.3in}
\setlength{\textheight}{8.7in}
\setlength{\topmargin}{0pt}
\setlength{\headsep}{0pt}
\setlength{\headheight}{0pt}
\setlength{\oddsidemargin}{0pt}
\setlength{\evensidemargin}{0pt}

% SET UP SECTION NUMBERING

\setcounter{secnumdepth}{1}
% Number sections, but not subsections

% BREAK THEOREM STYLE

\newtheoremstyle{break}% name
  {}%         Space above, empty = `usual value'
  {}%         Space below
  {}% Body font
  {}%         Indent amount (empty = no indent, \parindent = para indent)
  {\bfseries}% Thm head font
  {.}%        Punctuation after thm head
  {\newline}% Space after thm head: \newline = linebreak
  {}%         Thm head spec

% DEFINE THEOREM-LIKE STRUCTURES

\theoremstyle{plain}
\newtheorem{lemma}{Lemma}[section]           % L:
%\newtheorem{lemma}{Lemma}                    % L:
\newtheorem{proposition}[lemma]{Proposition} % P:
\newtheorem{theorem}[lemma]{Theorem}         % T:
\newtheorem{corollary}[lemma]{Corollary}     % C:
\newtheorem{conjecture}[lemma]{Conjecture}   % J:
\newtheorem{question}[lemma]{Question}       % Q:
\newtheorem{problem}[lemma]{Problem}         % B:

\newtheorem{observation}[lemma]{Observation} % O:
\newtheorem{remark}[lemma]{Remark}           % R:

\theoremstyle{definition}
\newtheorem{definition}[lemma]{Definition}   % D:
\newtheorem{example}[lemma]{Example}         % E:

\theoremstyle{break}
\newtheorem{algorithm}[lemma]{Algorithm}     % A:

% Also: Section = S:, Figure = Fig:, Item = It:, Equation = Eq:

% CHANGE APPEARANCE OF ENUMERATED LISTS

\renewcommand{\labelenumi}{(\roman{enumi})} % Labels (i), (ii), etc.

% CHANGE QED-RELATED COMMANDS

\renewcommand{\qed}{}
\newcommand{\ggcqedsymbol}{$\square$}
\newcommand{\ggcqed}{\hbox{}\nobreak\hbox{\quad\ggcqedsymbol}}
\newcommand{\ggcendpf}{\ggcqed}
%\newcommand{\ggcnopf}{}
\newcommand{\ggcnopf}{\ggcqed}
%\newcommand{\ggcendexample}{}
\newcommand{\ggcendexample}{\ggcqed}

% SET STYLE FOR DEFINED TERMS

\newcommand{\defterm}[1]{\emph{#1}} % Defined term
\newcommand{\abstdefterm}[1]{#1} % Defined term in abstract
\newcommand{\localdefterm}[1]{\emph{#1}} % Defined term used only nearby

% RUN-IN HEADINGS

\newcommand{\runinhead}[1]{\vskip0.1in\noindent\textbf{#1}} % Run-in heading
% This should be used at the start of a paragraph,
%  with no space between it and the first word of the paragraph.

% DEFINITIONS SPECIFIC TO THIS DOCUMENT

% (NONE)

\date{June 1, 2017}

\title{Path-Coloring Algorithms for Plane Graphs}

\author{Aven Bross\\
\small Department of Computer Science\\
\small University of Alaska\\
\small Fairbanks, AK 99775-6670\\
\small\texttt{bross{@}math.colostate.edu} \and
Glenn G.~Chappell\\
\small Department of Computer Science\\
\small University of Alaska\\
\small Fairbanks, AK 99775-6670\\
\small\texttt{chappellg{@}member.ams.org} \and
Chris Hartman\\
\small Department of Computer Science\\
\small University of Alaska\\
\small Fairbanks, AK 99775-6670\\
\small\texttt{cmhartman{@}alaska.edu}}

\begin{document}

\maketitle
\centerline{\small \textit{2010 Mathematics Subject Classification.}
 Primary 05C38; Secondary 05C10, 05C15.}
\centerline{\small \textit{Key words and phrases.}
 Path coloring, list coloring, algorithm.}

\begin{abstract}
A \abstdefterm{path coloring} of a graph $G$ is a vertex coloring
of $G$ such that each color class induces a disjoint union of paths.
We present two efficient algorithms
to construct a path coloring of a plane graph.

The first algorithm, based on a proof of Poh, %\cite{Poh1990}
is given a plane graph;
it produces a path coloring of the given graph
using three colors.

The second algorithm,
based on similar proofs
by Hartman % \cite{Har1997}
and \v{S}krekovski, %\cite{Skr1999}
performs a list-coloring generalization of the above.
The algorithm is given a plane graph and an assignment of lists of
three colors to each vertex;
it produces a path coloring of the given graph
in which each vertex receives a color from its list.

Implementations of both algorithms are available.
\end{abstract}


\section{Introduction}

All graphs will be finite, simple, and undirected.
See West~\cite{Wes2000} for graph theoretic terms.

A \defterm{path coloring} of a graph $G$ is a vertex coloring
(not necessarily proper) of $G$ such that each color class induces
a disjoint union of paths.
A graph $G$ is \defterm{path $k$-colorable} if $G$
admits a path coloring using $k$ colors.

Broere \& Mynhardt conjectured~\cite[Conj.~16]{BrMy1985}
that every planar graph is path $3$-colorable.
This was proven independently by Poh~\cite[Thm.~2]{Poh1990}
and by Goddard~\cite[Thm.~1]{God1991}.

\begin{theorem}[Poh 1990, Goddard 1991] \label{T:planar3c}
If $G$ is a planar graph,
then $G$ is path $3$-colorable.\ggcnopf\end{theorem}

It is easily shown that the ``$3$'' in Theorem~\ref{T:planar3c}
is best possible.
In particular, Chartrand \& Kronk~\cite[Section~3]{ChKr1969}
gave an example of a planar graph whose vertex set cannot be partitioned
into two subsets, each inducing a forest.

Hartman~\cite[Thm.~4.1]{Har1997}
proved a list-coloring generalization of Theorem~\ref{T:planar3c}
(see also Chappell \& Hartman~\cite[Thm.~2.1]{ChHa2017prep}).
A graph $G$ is \defterm{path $k$-choosable} if,
whenever each vertex of $G$ is assigned a list of $k$ colors,
there exists a path coloring of $G$ in which each vertex receives
a color from its list.

\begin{theorem}[Hartman 1997] \label{T:planar3}
If $G$ is a planar graph,
then $G$ is path $3$-choosable.\ggcnopf\end{theorem}

Essentially the same technique was used by
\v{S}krekovski~\cite[Thm.~2.2b]{Skr1999}
to prove a result slightly weaker than Theorem~\ref{T:planar3}.


\medskip

We discuss two efficient path-coloring algorithms
based on proofs of the above theorems.
We distinguish between a \defterm{planar} graph---one that
can be drawn in the plane without crossing edges---and
a \defterm{plane} graph---a graph with a given embedding
in the plane.

In Section~2 we outline our graph representations
and the basis for our computations of time complexity.

Section~3 covers an algorithm
based on Poh's proof of Theorem~\ref{T:planar3c}.
The algorithm is given a plane graph;
it produces a path coloring of the given graph
using three colors.

Section~4 covers an algorithm
based Hartman's proof of Theorem~\ref{T:planar3},
along with the proof of \v{S}krekovski mentioned above.
The algorithm is given a plane graph
and an assignment of a list of three colors to each vertex;
it produces a path coloring of the given graph
in which each vertex receives a color from its list.

Implementations of both algorithms are available;
see Bross~\cite{Bro2017}.


\section{Graph Representatons and Time Complexity}

We will represent a graph via \textit{adjacency lists}:
a list, for each vertex $v$, of the neighbors of $v$.
A vertex can be represented by an integer $0\dots n-1$,
where $n$ is the order of the graph.

A plane graph will be specified via
a \defterm{rotation scheme}:
a circular ordering,
for each vertex $v$, of the edges incident with $v$,
in the order they appear around $v$ in the plane embedding;
this completely specifies
the combinatorial embedding of the graph.
Rotation schemes are convenient when we represent a graph
using adjacency lists;
we simply order the adjacency
list for each vertex $v$ in clockwise order around $v$;
no additional data structures are required.

ZZZ Time Complexity ZZZ

ZZZ Augmented Adjacency Lists ZZZ

\section{Path Coloring: the Poh Algorithm}

ZZZ


\section{Path List Coloring: the Hartman-\v{S}krekovski Algorithm}

ZZZ


\begin{thebibliography}{99}

\bibitem{BrMy1985}
I.~Broere and C.~M.~Mynhardt,
Generalized colorings of outerplanar and planar graphs,
\textit{Graph theory with applications to algorithms and computer science}
 (Kalamazoo, Mich., 1984),
pp.~151--161,
Wiley-Intersci. Publ., Wiley, New York, 1985.

\bibitem{Bro2017}
A.~Bross,
\textit{Implementing path coloring algorithms on planar graphs},
Masters Project,
University of Alaska,
2017,
available from\hfil\break
\texttt{http://github.com/permutationlock/path\_coloring\_bgl}.

\bibitem{ChHa2017prep}
G.~G.~Chappell and C.~Hartman,
Path choosabiility of planar graphs,
in preparation.

\bibitem{ChKr1969}
G.~Chartrand and H.~V.~Kronk,
The point-arboricity of planar graphs,
\textit{J. London. Math. Soc.}
\textbf{44} (1969),
612--616.

\bibitem{God1991}
W.~Goddard,
Acyclic colorings of planar graphs,
\textit{Discrete Math.}
\textbf{91} (1991), no. 1,
91--94.

\bibitem{Har1997}
C.~M.~Hartman,
\textit{Extremal Problems in Graph Theory},
Ph.D. Thesis,
University of Illinois,
1997.

\bibitem{Poh1990}
K.~S.~Poh,
On the linear vertex-arboricity of a planar graph,
\textit{J. Graph Theory}
\textbf{14} (1990), no. 1,
73--75.

\bibitem{Skr1999}
R.~\v{S}krekovski,
List improper colourings of planar graphs,
\textit{Combin. Probab. Comput.}
\textbf{8} (1999), no. 3,
293--299.

\bibitem{Wes2000}
D.~B.~West,
\textit{Introduction to Graph Theory, 2nd ed.},
Prentice Hall,
Upper Saddle River, NJ,
2000.

\end{thebibliography}

\end{document}

